\documentclass[10pt]{article} % Use the res.cls style, the font size can be changed to 11pt or 12pt here

\usepackage{helvet} % Default font is the helvetica postscript font
\usepackage{hyperref}
\usepackage[a4paper,left=1in,top=1in,right=1in,bottom=1in]{geometry}
\usepackage{fancyhdr}

\pagestyle{fancy}
\lhead{Clement Fung}
\rhead{UC Berkeley Computer Science PhD Applicant}

\begin{document}

\begin{center}
{\large \bf Personal History Statement for UC Berkeley Computer Science PhD Application}
\end{center}

My path to applying to graduate school is an unconventional one: I completed my undergraduate degree in systems design engineering at the University of Waterloo, a program that focuses on systems modeling and product design. It was not until the summer of my sophomore year in 2013 when I was first exposed to research at Dr. Paul Boutros' bioinformatics lab at the University of Toronto. This was a particularly eye-opening experience for me as I got to work in an environment that was unlike industry internship positions I had held before; it intrigued me to see students helping and supporting each other in an academic environment, working on problems with completely unknown solutions. \\

I choose to expand my training in computer science through a masters degree in computer science at the University of British Columbia (UBC). This turned out to be pivotal in shaping my opinions of graduate school and choosing my subfield of computer science research, security and privacy in distributed machine learning. UBC was also where I was first introduced to research, focusing on problems in distributed systems security: I would eventually combine my experience in machine learning to produce three major research projects over my two year program. With my masters degree completed and my path clearer, I am now confident that a PhD in computer science is the ideal choice moving forward for my personal goals. \\

One particular element of my life that I feel especially gracious about is that several of my mentors have taken a chance on me and my growth. When I first interned at LinkedIn as an undergraduate student, I was the only intern who was not studying computer science or computer engineering. When my masters supervisor, Professor Ivan Beschastnikh, first approached me to be his student, I was merely a student in his course with minimal experience in research or building distributed systems. To be frank, I would not be where I am today without these two opportunities. Recognizing this, I have tried to do the same for others. When a particularly eager and inexperienced former student of mine told me that they were interested in computer science research, I strongly recommended them to my supervisor and they eventually became a co-author on two of my three research projects. Currently, they too are applying to graduate school in computer science. Providing opportunities and teaching others who are eager to learn has been an important value of mine throughout my life, and this translates into a great passion for teaching. I love to teach, and I was awarded a UBC Computer Science Graduate Student Teaching Assistant Award in 2017 for my efforts. My hope is that pursuing a PhD in computer science provide me with the ideal environment and opportunities to continue teaching and mentoring others.\\        

The academic journey is best enjoyed with others. When I started my graduate studies at UBC, I took it upon myself to become a champion of community, diversity and inclusiveness. I was elected as the president of the Computer Science Graduate Student Association (CSGSA) at UBC, this involved overseeing and organizing events for incoming students and serving as a liaison between students and faculty within the department. At UBC, most masters students are admitted without being matched with a research supervisor; they are instead given 8 months to find a research topic and a supervisor. One of my proudest achievements as president involved establishing a networking lunch for new incoming masters students and computer science research faculty. At this event, students and faculty would meet and discuss potential research topics in structured sessions; the event was attended by approximately 30 students and 12 faculty members and resulted in several supervisory relationships. Despite never working directly with a large majority of these students or faculty, I wanted to foster these relationships and help build a better sense of community amongst students and faculty. After my year as the CSGSA president ended, I was awarded a UBC Computer Science Department Service Award in 2017. \\

In addition to the educational and research benefits that a PhD provides, I am especially drawn to pursuing a PhD in computer science for the academic community. I love collaborating with great people on engaging challenges, and I was fortunate enough to meet and learn from some superb researchers while at UBC. I cannot think of a better environment than UC Berkeley for a PhD, which brings together such students from a diverse set of educational and cultural backgrounds and allows them to challenge the frontier of research. PhDs are great networking tools for meeting inquisitive, motivated people and results in a lifelong network of friendships and collaborations. To me, this is what truly matters: the academic experience is best when shared with a diverse, tight-knit community with a culture of shared, reciprocal mentorship. 

\end{document}
