\documentclass[10pt]{article} % Use the res.cls style, the font size can be changed to 11pt or 12pt here

\usepackage{helvet} % Default font is the helvetica postscript font
\usepackage{hyperref}
\usepackage[a4paper,left=1in,top=1in,right=1in,bottom=1in]{geometry}

\begin{document}

\begin{center}
{\Large \bf Clement Fung - Personal History Statement for UC Berkeley Computer Science PhD Application}
\end{center}

% Many others have helped me: Linkedin, Ivan. I have started to give back

My path to applying to graduate school at UC Berkeley was an unconventional one, with many changes and steps along the way. I completed my undergraduate degree in systems design engineering at the University of Waterloo, a program that focuses on systems modeling, product design and human factors engineering. Computer science and computer science research were not major aspects of this program. It was not until the summer of my sophomore year in 2013 when I was first exposed to research at Dr. Paul Boutros' bioinformatics lab at the University of Toronto. This was a particularly eye-opening experience for me as I got to work in an environment that was unlike industry internship positions I had held before; it intrigued me to see students helping and supporting each other in an academic environment, working on problems with completely unknown solutions. \\

From this point onwards, pursuing an academic career was at the back of my mind. After finishing my undergraduate degree, which included three additional 4 month internships, I decided to try graduate school. I choose to expand my training in computer science through a masters degree in computer science at the University of British Columbia (UBC). This turned out to be an excellent choice: my time at UBC was pivotal in shaping my opinions of graduate school and choosing my subfield of research in computer science. I was first introduced to research at UBC and even focused on the research area of distributed systems security: I would eventually combine my experience in machine learning to produce three major research projects over my two year program. With my masters degree completed and my path clearer, I am now confident that a PhD in computer science is the ideal choice moving forward. \\

In addition to the educational and research benefits that a PhD provides, I am especially drawn to pursuing a PhD in computer science after enjoying the benefits of the academic community at UBC. I love collaborating with great people on engaging challenges, and I was fortunate enough to meet and learn from some superb researchers while at UBC. I cannot think of a better environment than UC Berkeley for a PhD, which brings together such students from a diverse set of educational and cultural backgrounds and allows them to challenge the frontier of research. PhDs are great networking tools for meeting inquisitive, motivated people and results in a lifelong network of friendships and collaborations. \\

The academic journey is best enjoyed with others. When I started my graduate studies at UBC, I took it upon myself to become a champion of community, diversity and inclusiveness. I was elected as the president of the Computer Science Graduate Student's Association CSGSA) at UBC, this involved overseeing and organizing events for incoming students and serving as a liaison between students and faculty within the department. At UBC, many masters students are admitted without being matched with a research supervisor; they are given 8 months to find a research topic and a supervisor. One of my proudest achievements as president involved establishing a networking lunch for new incoming masters students and computer science research faculty. At this event, students and faculty would meet and discuss potential research topics; the event was attended by approximately 30 students and 12 faculty members, and resulted in several supervisory relationships. Despite never working directly with a large majority of these students or faculty, I found that the relationships formed helped foster a sense of community amongst students and faculty, once that continued forwards through the academic year. \\

To me, this is what truly matters: the academic experience is best when shared with a diverse, tight-knit community. At UC Berkeley, I hope to experience and foster such a community. In addition to its research excellence, the diverse community of the Bay Area and the student body at UC Berkeley are major draws for me towards applying for a PhD here. All in all, I have loved being involved in the academic community, and I would love nothing more than to continue growing alongside those around me. \\


\end{document}
